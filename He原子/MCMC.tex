\documentclass{article}
\usepackage{amsmath}
\usepackage{ctex}

\begin{document}

\section{Metropolis算法}

在这一节中,我们提出了一种实际的方法来构造一个满足详细平衡条件(式3.45)的条件概率 $\omega\left(x^{\prime} \mid x\right)$,使得对于大值的 $n$,配置 $x_{n}$ 按照给定的概率分布 $\mathcal{P}_{\mathrm{eq}}(x)$ 分布。Metropolis及其合作者(Metropolis等人,1957年)引入了一种非常简单的方案,也非常通用,可以应用于许多不同的情况。后来,所谓的Metropolis算法被W. Keith Hastings(1970年)扩展到更一般的情况(经常也使用“Metropolis-Hastings算法”的名称)。作为第一步,我们将过渡概率 $\omega\left(x^{\prime} \mid x\right)$ 分成两部分:

\[
\omega\left(x^{\prime} \mid x\right)=T\left(x^{\prime} \mid x\right) A\left(x^{\prime} \mid x\right),
\]

其中 $T\left(x^{\prime} \mid x\right)$ 定义了一个试验概率,从当前配置 $x$ 中提出新的配置 $x^{\prime}$,$A\left(x^{\prime} \mid x\right)$ 是接受概率。在Metropolis及其合作者的原始工作中,试验概率被假定为对称的,即 $T\left(x^{\prime} \mid x\right)=T\left(x \mid x^{\prime}\right)$。然而,在算法的广义版本中,只要确保遍历性,就可以选择具有很大自由度的 $T\left(x^{\prime} \mid x\right)$。然后,为了定义满足详细平衡条件的马尔可夫过程,接受提议的配置 $x^{\prime}$ 的概率为:

\[
A\left(x^{\prime} \mid x\right)=\operatorname{Min}\left\{1, \frac{\mathcal{P}_{\mathrm{eq}}\left(x^{\prime}\right) T\left(x \mid x^{\prime}\right)}{\mathcal{P}_{\mathrm{eq}}(x) T\left(x^{\prime} \mid x\right)}\right\}.
\]

不失一般性,我们总是可以选择 $T(x \mid x)=0$,即我们永远不提议保持相同的配置。然而,$\omega(x \mid x)$ 可能是有限的,因为提议的移动可能会被拒绝。 $\omega(x \mid x)$ 的实际值由归一化条件确定:$\sum_{x^{\prime}} \omega\left(x^{\prime} \mid x\right)=1$。

在大多数情况下(如Metropolis及其合作者的原始工作中),考虑对称试验概率 $T\left(x^{\prime} \mid x\right)=T\left(x \mid x^{\prime}\right)$ 是有用的。在这种情况下,接受概率简化为:

\[
A\left(x^{\prime} \mid x\right)=\operatorname{Min}\left\{1, \frac{\mathcal{P}_{\mathrm{eq}}\left(x^{\prime}\right)}{\mathcal{P}_{\mathrm{eq}}(x)}\right\}。
\]

\end{document}
